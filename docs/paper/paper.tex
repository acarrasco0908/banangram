\documentclass[twocolumn]{revtex4}

\usepackage{amsfonts}
\usepackage{amssymb}
\usepackage{amsmath}
\usepackage{graphicx}
\usepackage{verbatim}
\usepackage{epstopdf} %For EPS figures
\usepackage[colorlinks=False,hidelinks=True]{hyperref}
\usepackage[colorinlistoftodos]{todonotes} %Todo notes
\usepackage{cleveref} %Multiple refs in one tag
\usepackage{tikz}
\usepackage{caption}
\usepackage{subcaption}
\captionsetup{compatibility=false}

\begin{document}

\title{Random Bananagrams}

\author{Colin Clement}
\affiliation{Cornell University}
\author{Samuel Kachuck}
\affiliation{Cornell University}

\date{\today}

\begin{abstract}
\end{abstract}

\maketitle

\section{Introduction}

\section{Algorithm}

\section{Theoretical Stuff}
\subsection{Probability of a word in a rack}
Given a lexicon, a character frequency, and a random rack of $n$ tiles, $R_n$, what is the probability that a word $w_k$ from the lexicon can be made from those tiles. $w_k$ is made up of $k$ letters, whose (independent) probabilities of being drawn are $P(l_0),\ldots,P(l_{k-1})$.

$P(w_k \in R_n)$

\begin{equation}
P(w_k \in R_k) = \frac{k!}{\prod_{c\in w_k} m_c} P(l_0)\cdot P(l_{k-1}),
\end{equation}
with $m_c$ the number of occurences of character $c$ in the word (i.e., in the word `feet', $m_\text{e} = 2$, $m_\text{f}=1$, etc.).

Now, the probability that 


\section{Science!}
\subsection{Lexicon Results}
\emph{Hypothesis 1}: As the rack size increases, the probability of finding a solution should limit to 1 (a solution exists).

The solid blue line in figure \ref{fig:solbyrack} demonstrates that this is not so, as the proportion of racks that terminate in a solution limits to about 96\% for the twl06 lexicon. Figure \ref{fig:branchlimhist} indicates that the issue is not in hitting a branch limit, but rather confirms that these are boards without solutions (some have been checked by hand).

\emph{Hypothesis 2}: This limiting value is a property of the lexicon and the far more permitting lexicon sowpods will limit to 1.

The solid red line in figure \ref{fig:solbyrack} shows that this, too, is not the case. In fact, the two lexica appear to have the same limiting value.

\end{document}
